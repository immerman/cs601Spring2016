\documentclass[12pt]{article}
\usepackage{defs601}                   
\hwstyle
\begin{document}
\noindent\framebox[\textwidth]{
{\normalsize CS601:} \hfill {\bf Homework 1} \hfill 
{Due in class Thursday, Jan. 21, 2016}}
\addtocounter{section}{1}

Remember, you should solve these problems, so that you are able to discuss or present the solutions
in the next class.  You do not need to write these up, but it is a good idea to write down
notes for yourself from which you can quickly and easily generate the solutions.

We will think of a Turing Machine input as a binary string, $w\in \set{0,1}^*$, which appears
between the left marker, ``$\triangleright$'' and the infinite string of blanks, $\sqcup$.  The
following problems show that we can ignore the type of the input:  if we want to think of the input
as a binary string, $w$, that's what it is.  If we want to think of it as a natural number, fine,
it's $\zeta(w)$.  If we want to think
of it as a pair of natural numbers, fine it's $(L(\zeta(w)), R(\zeta(w)))$, etc.

\begin{enumerate}
\item Show that the following pairing function given by Cantor is a 1:1
correspondence between $\N\times\N$ and $\N$:
\[ P(i,j) \quad\mapsto\quad {(i+j)(i+j+1)\over 2} + i\]
[Hint: look at enough examples to get an understanding of what this
function does.  Then give a clear, simple proof that this map is indeed 1:1
and onto.  I don't care how formal you are as long as your argument is very clear
and convincing.  It may help to recall that $\sum_{i=1}^n i = n(n+1)/2$.
You might want to show the existence of the ``inverses'' of the pairing function.  These are
functions $L,R: \N\rightarrow \N$ with the properties that
\[ \forall i,j,k \in \N \,( P(L(k),R(k)) = k \quad \&\quad L(P(i,j)) = i \quad \&\quad R(P(i,j))= j)\]
\item Construct a simple, easy to compute 1:1 and onto map, $\zeta$, from the set of binary strings
  to the naturals, i.e., $\zeta: \set{0,1}^\star\bijection\N$.
\end{enumerate}

\end{document}

