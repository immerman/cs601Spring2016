\documentclass[12pt]{article}
\usepackage{defs601}                   
\hwstyle
\begin{document}
\thispagestyle{empty}
\noindent\framebox[\textwidth]{
{\normalsize CS601:} \hfill {\bf Homework 9} \hfill 
{Due in class Tue, Mar.\ 8, 2016}}
\addtocounter{section}{9}

As ususal, please ask on Piazza if parts of the following two problems need hints or clarifications.

\begin{enumerate}

\item I believe that, in some ways, the complexity class NP includes most creative work that
people do, or at least that part of it that can be put on a CD when it is
done.  As an example, argue that the following set is NP-complete:

\[ \mbox{MATH}\qe \bigset{\phi \#^n}{\mbox{ZFC}\proves\phi \mbox{ and the
proof has length } \leq n \mbox{ characters}}\]

Here ``$\mbox{ZFC}\proves\phi$'' means that there is a formal proof of $\phi$ from the standard
axiomatization 
of set theory, namely the Zermelo-Fraenkel axioms plus the axiom of choice.  However, especially as
we haven't talked about this, just assume that we have some simple but reasonably powerful
proof system that is machine readable and checkable.  As usual, to prove that MATH is NP complete
you should show that it is in NP, and then provide a reduction from 3SAT to MATH.


\item The class $\thc^0$ is a rather interesting complexity class.  Show
that the following arithmetic operations are computable in $F(\thc^0)$:
\begin{enumerate}
\item The sum of two $n$-bit natural numbers.  [Hint: see notes from Descriptive Complexity Lecture]
\item The number of ``1''s in a binary string of length $n$.
\item The sum of $n$ $n$-bit natural numbers: $C_1 + \cdots + C_n$

[Major Hint: break each $C_i$ into stripes: $C_i = A_i + B_i$ where $A_i$
has its first, third, fifth, etc.,  $\log n$ bits 0, $B_i$ has its even
groups of $\log n$ bits 0.  Then add up the $A_i$'s and the
$B_i$'s.  Finally just add the two sums using (a). The advantage of this is
that we only have to look at $\log n$ columns at a time to compute all the
carries.  In this way, you can use (b) to  reduce the problem of adding $n$ $n$-bit
numbers, to the problem of adding $\log n$ $\log(n)$-bit numbers.  

Now, do this same reduction again: all you have to do is add $\log \log n$
$\log \log n$-bit numbers.  This last problem is so small that you can
build a gate for each possible sum and then just determine which one of
them has the correct input.]
\item Multiplication of two $n$-bit integers.
\item Multiplication of two $n\times n$ integer matrices, each of whose
entries is an integer of at most $n$ bits.
\item It follows that multiplication of n $n\times n$ integer matrices, each of whose
entries is an integer of at most $n$ bits can be done in  $\thc^1$.  Why?
It turns out that almost all problems in linear algebra are reducible to
this problem and thus doable in $\thc^1$.
\end{enumerate}


\end{enumerate}
\end{document}

