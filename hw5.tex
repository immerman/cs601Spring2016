\documentclass[12pt]{article}
\usepackage{defs601}                   
\hwstyle
\begin{document}
\thispagestyle{empty}
\noindent\framebox[\textwidth]{
{\normalsize CS601:} \hfill {\bf Homework 5} \hfill 
{Due in class Thr, Feb. 4, 2016}}
\addtocounter{section}{5}

\begin{enumerate}
\item[0] We didn't finish going over two the proofs from Theorem 4.2 from last time -- EMPTY-CFL and
MEMBER-CFL.  Please be prepared to talk about these next time.  Also please do the following problems:

\[
\begin{array}{lrcl}
\textbf{0a.}&\mbox{{\rm EmptyCFL}} &=& \bigset{G}{G \mbox{ is a CFG};\; \sel(G) =
  \emptyset} \\[1ex] 
\textbf{0b.}&\mbox{{\rm MemberCFL}} &=& \bigset{\angle{G,w}}{G \mbox{ is a CFG};\;w\in\sel(G)}\\
\end{array}
\]


\item   Recall the definition of $F(\dtime[t(n)])$ from the slides for lecture 5.  In particular,
  $F(\p)$ is the set of total functions computable by Turing machines that run in polynomial time.  

Consider the set PTM  
$=\bigset{M_i}{\exists c \forall n(\mbox{ TM $M_i$ runs in time $cn^c$ on all inputs of length $n$}
  )}$. 
That is, PTM is the set of TMs that run in polynomial time.
Give two reductions to show that PTM is neither r.e.\ nor co-r.e.

\item Show that there is a recursive set of TMs, $R \subseteq$ PTM, such that for all $f \in F(\p)$,
  there exists $M_i \in R$ such that the function computed by $M_i$ is $f$.  [Hint:  construct $R$
    as the set of all TMs which run with an explicit polynomial-time clock which shuts the TM off
    when the clock expires.]
\end{enumerate}

\end{document}

