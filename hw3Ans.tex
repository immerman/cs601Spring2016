\documentclass[12pt]{article}
\usepackage{defs601}
\hwstyle
\begin{document}
\noindent\framebox[\textwidth]{
{\normalsize CS601:} \hfill {\bf Answers to hw3} \hfill 
{Spring 2016}}
\addtocounter{section}{1}

Prove that the following sets are r. e. complete:
\begin{eqnarray*}
A_{\text{TM}} &=& \bigset{(i, j)}{M_i(j) = 1} \\[2ex]
\halt &=& \bigset{(n, m)}{M_n(m) \mbox{ halts}} \\[2ex]
\ov{\notfull} &=& \bigset{n}{W_n \neq \emptyset}
\end{eqnarray*}

\begin{enumerate}
\item Show $A_{\text{TM}}$ is r. e. complete.

$A_{\text{TM}}$ is r. e., because on input $(i, j)$ we can simulate $M_i$ on input $j$, and accept if $M_i$ accepts.

To show that $A_{\text{TM}}$ is r. e. hard, we reduce $K = \bigset{i}{M_i(i) = 1}$ to $A_{\text{TM}}$. As $K$ is r. e. hard, this shows $A_{\text{TM}}$ is r. e. hard as well.

Let the reduction $f$ be the function such that $f(n) = (n, n)$.

Then $n \in K \iff M_n(n) = 1 \iff (n, n) \in A_{\text{TM}} \iff f(n) \in A_{\text{TM}}$.

As $A_{\text{TM}}$ is r. e. and r. e. hard, it is r. e. complete.

{\bf Answer by: Larkin Flodin} \quad {\bf with edits by:}

\item Show $\halt$ is r. e. complete.

$\halt$ is r. e., because on input $(i, j)$ we can simulate $M_i(j)$ and accept if $M_i$ ever halts.

For r. e. hardness, we reduce $K$ to $\halt$. Let $f(n) = (g(n), 0)$, where $g(n)$ is the index of the machine $M_{g(n)}$ that ignores its input, and simulates $M_n(n)$. If $M_n$ accepts, then $M_{g(n)}$ halts, otherwise if $M_n$ halts without accepting, $M_{g(n)}$ goes into an infinite loop.

Then $n \in K \iff M_n(n) = 1 \iff M_{g(n)} \textrm{ halts on all inputs} \iff M_{g(n)}(0) \textrm{ halts} \iff (g(n), 0) \in \halt \iff f(n) \in \halt$.

As $\halt$ is r. e. and r. e. hard, it is r. e. complete.

{\bf Answer by: Larkin Flodin} \quad {\bf with edits by:}

\item Show $\ov{\notfull}$ is r. e. complete.

$\ov{\notfull}$ is r. e., because on input $n$ we can dovetail the simulation of $M_n$ on all possible inputs, simulating one step of computation on the first input, then two steps on the first two inputs, and so on. If $M_n$ ever accepts an input, then accept.

We reduce $\ov{\notfull}$ to $A_{0, 17} = \bigset{i}{M_i(0) = 17}$ to show r. e. hardness.

Let $f$ be the function such that $f(n)$ is the index of the machine that ignores its input, and simulates $M_n(0)$. If $M_n$ outputs 17, then $M_{f(n)}$ accepts, otherwise $M_{f(n)}$ rejects.

Then $n \in A_{0, 17} \iff M_n(0) = 17 \iff \exists x, M_{f(n)}(x) = 1 \iff W_{f(n)} \neq \emptyset \iff f(n) \in \ov{\notfull}$.

As $\ov{\notfull}$ is r. e. and r. e. hard, it is r. e. complete.

{\bf Answer by: Larkin Flodin} \quad {\bf with edits by:}

\end{enumerate}

\end{document}