\documentclass[12pt]{article}
\usepackage{defs601}                   
\hwstyle
\begin{document}
\thispagestyle{empty}
\noindent\framebox[\textwidth]{
{\normalsize CS601:} \hfill {\bf Homework 7} \hfill 
{Due in class Thr, Feb. 18, 2016}}
\addtocounter{section}{7}

Please make sure that you have completed the solutions to the homework problems that you are
responsible for.   To start getting ready for the midterm, March 1, it would be good for everyone to
go over all these solutions and make sure that you understand them.  (Helping with corrections and
proof reading would be appreciated.)

Show that the following three problems are all NL complete:

\textbf{Answers by: Dan Saunders}

\begin{enumerate}
\item  EMPTY-DFA $ = \bigset{D}{D \mbox{ a deterministic finite automaton and } \sel(D) =
  \emptyset}$ 

To show that $EMPTY-DFA$ is $NL$-complete, we must show that $EMPTY-DFA$ is both in $NL$, and for all languages $A$ $\in$ $NL$, $A$ $\leq$ $EMPTY-DFA$. We provide a reduction from $REACH$, an $NL$-complete language itself, to $\overline{EMPTY-DFA}$, taking advantage of the fact that $NL = coNL$, which will prove both of these conditions. The reduction will take instances of $REACH$ into instances of $\overline{EMPTY-DFA}$, and is as follows: 

We let node $s$ be the start state of the graph instance of $\overline{EMPTY-DFA}$, and $t$ the only final state in the graph. The vertices and edges of the $REACH$ instance are preserved in the graph of the instance of $\overline{EMPTY-DFA}$. Now, clearly this reduction takes space logarithmic in the size of the $REACH$ instance, as we are simply copying over the graph to the instance of $\overline{EMPTY-DFA}$. Now, this reduction will take tuples $<G, s, t>$ to a $DFA$ $D$ such that $L(D) \neq \emptyset$ iff there exists a path from $s$ to $t$ in $G$, and therefore, $REACH \leq \overline{EMPTY-DFA} \implies REACH \leq EMPTY-DFA \implies EMPTY-DFA$ is $NL$-complete, given that we may use the $NL REACH$ algorithm to decide $EMPTY-DFA$.

\vspace*{.1in}

\item EMPTY-NFA $ = \bigset{N}{N \mbox{ a nondeterministic finite automaton and } \sel(N) =\emptyset}$

This construction is exactly analogous to the first problem, and so we may immediately conclude that $EMPTY-NFA$ is $NL$-complete.

\vspace*{.1in}

\item EQUAL-DFA $ =  \bigset{(D_1,D_2)}{D_1,D_2 \mbox{ DFAs};\;\sel(D_1) =\sel(D_2)}$

It is clear that $< D_1$, $D_2 >$ is in $EQUAL-DFA$ if and only if the symmetric difference of $D_1$, $D_2$ is in $EMPTY-DFA$, and so, $EQUAL-DFA \ in NL$, since $EMPTY-DFA \in NL$ and $EQUAL-DFA \leq EMPTY-DFA$.

We now show that $EQUAL-DFA$ is $NL$-hard by reducing the $NL$-complete language $EMPTY-DFA$ to $EQUAL-DFA$ as follows:

The reduction takes some $DFA$ $D$ to the tuple $< D, D'>$, where $L(D') = \emptyset$, therefore, $D \in EMPTY-DFA$ if and only if $<D, D'> \in EQUAL-DFA$. We conclude that $EQUAL-DFA$ is $NL$-complete.

\end{enumerate}
\end{document}
