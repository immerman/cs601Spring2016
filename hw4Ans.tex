\documentclass[12pt]{article}
\usepackage{defs601}                   
\hwstyle
\begin{document}
\noindent\framebox[\textwidth]{
{\normalsize CS601:} \hfill {\bf Answers to hw4} \hfill 
{Spring 2016}}
\addtocounter{section}{1}

\begin{enumerate}
\item Prove that if $S$ is r.e. hard or co-r.e. hard, then $S$ is not recursive. (From now on, the main way we will
prove that a problem is not recursive, i.e., not computable, is to show that it is r.e. hard or co-r.e. hard.)

{\bf Answer by:} \quad {\bf with edits by:}

\item Prove Theorem 4.2 from the posted notes for Lecture 4.
\[
\begin{array}{lrcl}
\textbf{2a.}&\mbox{{\rm EmptyNFA}} &=& \bigset{N}{N \mbox{ is an NFA};\; \sel(N) = \emptyset} \\[1ex]
\textbf{2b.}&\Sigma^\star\mbox{{\rm DFA}} &=& \bigset{D}{D\mbox{ is a DFA};\;
  \sel(D) = \Sigma^\star}\\[1ex]
\textbf{2c.}&\mbox{{\rm MemberNFA}} &=& \bigset{\angle{N,w}}{N \mbox{ is an NFA};\; w\in \sel(N)}\\[1ex]
\textbf{2d.}&\mbox{{\rm EqualDFA}} &=& \bigset{\angle{D_1,D_2}}{D_1,D_2 \mbox{ DFAs};\;\sel(D_1) =\sel(D_2)}\\[1ex]
\textbf{2e.}&\mbox{{\rm EmptyCFL}} &=& \bigset{G}{G \mbox{ is a CFG};\; \sel(G) =
  \emptyset} \\[1ex] 
\textbf{2f.}&\mbox{{\rm MemberCFL}} &=& \bigset{\angle{G,w}}{G \mbox{ is a CFG};\;w\in\sel(G)}\\
\end{array}
\]

\begin{itemize}
\item \textbf{2d} There are two algorithms that we can use to determine if two DFA are in EqualDFA. 

\textbf{Option 1} is to convert both DFAs to their minimal representation, and then compare the resulting minimal represenations isomorphically to determine if they are equivalent. 

In order to build the minimal representation, or standard automaton, for a DFA, use the following algorithm:

\[
\begin{array}{l}
\textrm{Let $D_i = (Q, \Sigma, \delta, s, F)$ be a DFA} \\
\textrm{Define a relation $A_{D_i} \subseteq K \times \Sigma*$ as:} \\
\quad (q,w) \in A_{D_i} \iff (q,w) \vdash_{D_i}* \textrm{ for some } f \in F \\
\textrm{Group all \textit{equivalent states} together.} \\
\quad \textrm{Two states $p,q \in Q$ are equivalent iff } \forall z \in \Sigma*: (q,z) \in A_{D_i} \iff (p,z) \in A_{D_i} \\
\end{array}
\]

\textbf{Option 2} is to check if the intersection of $D_1$ and $D_2$'s complement is $\emptyset$ and the intersection of $D_1$'s complement and $D_2$ is $\emptyset$. More formally:

\[
\begin{array}{lcr}
L(D_1 \cap \overline{D_2}) & = & \emptyset \\
L(\overline{D_1} \cap D_2) & = & \emptyset \\
\end{array} 
\]

The intersection of two DFAs $D_1 = (Q_1, \Sigma_1, \delta_1, s_1, F_1)$ and $D_2 = (Q_1, \Sigma_2, \delta_2, s_2, F_2)$ is (assuming $\Sigma+1 = \Sigma_2: $

\begin{equation*}
D_{1\cap 2} = (Q_1 \times Q_2, \Sigma, \delta_{1\cap 2}, (s_1,s_2), F_{1\cap 2})
\end{equation*}

Where $F_{1\cap 2}$ is the set of end states $(f_1,f_2)$ where $f_1 \in F_1$ and $f_2 \in F_2$ and $\delta_{1\cap2}$ applies the partial transition functions from each original DFA to generate a new combined state. Once $D_1 \cap \overline{D_2}$ and $\overline{D_1} \cap D_2$ are generated, you can use EmptyNFA above to determine if $L(D_1 \cap \overline{D_2}) = \emptyset$ and $L(\overline{D_1} \cap D_2) = \emptyset$. If both are true, then $(D_1,D_2) \in EqualDFA$.

\item \textbf{2e} The following algorithm will determine if $L(G) = \emptyset$ for a given CFG $G = (V,\Sigma, R, S)$:

\[
\begin{array}{l}
\textrm{Mark each terminal symbol in }G \\
\textrm{Do:} \\
\quad \textrm{Mark any symbol $A$ where there is a rule in $G$ such that }A \rightarrow w_1w_2...w_k \\
\quad \textrm{ where each symbol in $w_1w_2...w_k$ has already been marked.} \\
\textrm{Until no new symbols get marked.} \\
\textrm{If S is not marked: accept. Else: reject.}
\end{array}
\]

\item \textbf{2f} The following algorithm will determine if $(G,w) \in MemberCFL$, that is,

\begin{equation}
\{\langle G,w\rangle\ | G \in CFG, w \in L(G)\}
\end{equation}

for a given CFG $G = (V, \Sigma, R, S)$ and a given string $w$ with length $n = |w|$. 

\[
\begin{array}{l}
\textrm{Convert $G$ into its Chomsky Normal Form (CNF) representation, $G'$} \\ \\
\textrm{Let $N[i,j]$ be the set of all symbols in $V$ that can derive in $G$ the string $x_i...x_j$, with $i < j$.} \\
\textrm{for $i = 1$ to $n$ do}: N[i,i] = \{w_i\} \\
\textrm{for $j = 1$ to $n-1$ do}: \\
\quad \textrm{for $i = 1$ to $n-j$ do}: \\
\quad\quad \textrm{for $k = 1$ to $i+j-1$ do}: \\
\quad\quad\quad \textrm{if there is a rule} A \rightarrow BC \in R \textrm{where} B \in N[i,k] \textrm{and} C \in N[k+1,i+j]: \\
\quad\quad\quad\quad \textrm{then add $A$ to} N[i, i+j]\\
\textrm{if } S \in N[i,n] \textrm{: accept. Else: reject.} 
\end{array}
\]

\end{itemize}
{\bf Answer by: John Lalor} \quad {\bf with edits by:}

\end{enumerate}

\end{document}
