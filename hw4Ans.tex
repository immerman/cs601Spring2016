\documentclass[12pt]{article}
\usepackage{defs601}                   
\hwstyle
\begin{document}
\noindent\framebox[\textwidth]{
{\normalsize CS601:} \hfill {\bf Answers to hw4} \hfill 
{Spring 2016}}
\addtocounter{section}{1}

\begin{enumerate}

\item Prove that if $S$ is r.e. hard or co-r.e. hard, then $S$ is not recursive. (From now on, the main way we will
prove that a problem is not recursive, i.e., not computable, is to show that it is r.e. hard or co-r.e. hard.)


{\bf Answer by: Dan Saunders} \quad {\bf with edits by: Neil}

\textbf{1)} Let a set $S$ be r.e. hard.
  Then for all sets $T \in$ r.e., $T \leq S$. But, we know that the diagonal set $K \in$
  r.e., and therefore, $K \leq S$. We have already shown that $K$ is not recursive.
  Therefore, by the Fundamental Theorem of Reductions (Lecture 4), 
  $S$ is not recursive  either. 
  
  Similarly, if a set $S$ is co-r.e. hard, it
  cannot be recursive, as we find that $\ov{K} \leq S$, and since $\ov{K}$ is not recursive,
  neither is $S$.


\item Prove Theorem 4.2 from the posted notes for Lecture 4.  The following 6 sets are all in P.
\[
\begin{array}{lrcl}
\textbf{2a.}&\mbox{{\rm EmptyNFA}} &=& \bigset{N}{N \mbox{ is an NFA};\; \sel(N) = \emptyset} \\[1ex]
\textbf{2b.}&\Sigma^\star\mbox{{\rm DFA}} &=& \bigset{D}{D\mbox{ is a DFA};\;
  \sel(D) = \Sigma^\star}\\[1ex]
\textbf{2c.}&\mbox{{\rm MemberNFA}} &=& \bigset{(N,w)}{N \mbox{ is an NFA};\; w\in \sel(N)}\\[1ex]
\textbf{2d.}&\mbox{{\rm EqualDFA}} &=& \bigset{(D_1,D_2)}{D_1,D_2 \mbox{ DFAs};\;\sel(D_1) =\sel(D_2)}\\[1ex]
\textbf{2e.}&\mbox{{\rm EmptyCFL}} &=& \bigset{G}{G \mbox{ is a CFG};\; \sel(G) =
  \emptyset} \\[1ex] 
\textbf{2f.}&\mbox{{\rm MemberCFL}} &=& \bigset{(G,w)}{G \mbox{ is a CFG};\;w\in\sel(G)}\\
\end{array}
\]

{\bf Answer by: John Lalor} \quad {\bf with edits by: Neil}

\begin{itemize}

\item \textbf{2a)} Let $N$ be an arbitrary NFA. In order to decide whether $\sel(N) = \emptyset$, we build a machine that does the following:

\begin{enumerate}

\item We begin by checking whether the start node of $N$ is final. If so, we reject. Otherwise, we put a mark on it and add each node to which we can travel to in one step to a list. 

\item For each node in the list, if it is a final state, reject. Otherwise, we mark the node and add each unmarked node to which we can travel to in one step to the list, and then remove the current node from the list.

\item We repeat (b) until the list is empty, and accept if and only if we have not seen any nodes that are final states of $N$. 

\end{enumerate}

This search operates in time linear in the number of nodes in the NFA, and so we have that EmptyNFA $\in P$.

\item \textbf{2b)} Let $D$ be an arbitrary DFA. In order to determine whether ${\sel(D) = \Sigma^\star}$, we build a machine that performs the following procedure:

\begin{enumerate} 

\item Begin by marking the start state of $D$, and add each node to which we can travel to in one step to a list.

\item For each node in the list, if it is a non-final state, reject. Otherwise, mark the node and add each unmarked node to which we may travel to in one step to the list, and then remove the current node from the list.  

\item Repeat this process until the list is empty, and accept if and only if we have not seen any nodes that are not in the set of final states, $F$.

\end{enumerate}

Since this search operates in linear time with respect to the number of nodes in the DFA, we have
shown that $\Sigma^\star\mbox{{\rm DFA}}$ $\in$ $P$.  [{\color{blue} Note from Neil: another solution
    is to let $\ov{D}$ be the modification of $D$ in which the final and non-final states are
    switched.  Then $D\in \Sigma^\star\mbox{{\rm DFA}}$ iff $\ov{D} \in $ EmptyNFA, so by 2a., 
   $\Sigma^\star\mbox{{\rm DFA}}\in \p$.}]

\item \textbf{2c)} Let $N$ be an arbitrary NFA, and let $w$ be an arbitrary input to $N$. Then, the following algorithm will check whether $w \in \sel(N)$:

\begin{enumerate}

\item We begin by creating a list which will store the possible positions of the NFA $N$ at each step in its computation. We add the start state to the list, and any states that can be reached via an $\epsilon$-move.

\item For each character in the string $w$, we perform a step of the computation as follows: We create a new, empty list, and for each state in the old list, we place the set of states (to be moved to in one step) specified by the transition function of $N$ in accordance with both the current state and the character currently being read, including possible $\epsilon$-moves. 

\item At the end of this computation, we have a list of states in which the machine could be after performing its computation. We accept if any of the states are final states, and reject otherwise.

\end{enumerate}

This algorithm is clearly polynomial in the length of the input, $(N,w)$, and so MemberNFA $\in P$.

\item \textbf{2d)} There are two algorithms that we can use to determine if a pair of DFAs is in EqualDFA. 

\textbf{Option 1} is to convert both DFAs to their minimal representation, and then compare the
resulting minimal represenations to see if they are identical.  


\textbf{Option 2} is to check if the intersection of $D_1$ and $D_2$'s complement is $\emptyset$ and the intersection of $D_1$'s complement and $D_2$ is $\emptyset$. More formally:

\[
\begin{array}{lcr}
L(D_1 \cap \overline{D_2}) & = & \emptyset \\
L(\overline{D_1} \cap D_2) & = & \emptyset \\
\end{array} 
\]

The intersection of two DFAs $D_1 = (Q_1, \Sigma, \delta_1, s_1, F_1)$ and $D_2 = (Q_2, \Sigma,
\delta_2, s_2, F_2)$  is

\begin{equation*}
D_{1\cap 2} = (Q_1 \times Q_2, \Sigma, \delta_{1\cap 2}, (s_1,s_2), F_{1\cap 2})
\end{equation*}

Where $F_{1\cap 2}$ is the set of end states $(f_1,f_2)$ where $f_1 \in F_1$ and $f_2 \in F_2$ and
$\delta_{1\cap2}$ applies the transition functions from each original DFA, componentwise, i.e.,
\[ \delta_{1\cap2}((q_1,q_2),a) \quad \eqdef \quad (\delta_1(q_1,a), \delta_2(q_2,a))\; .\]
Once $D_1 \cap \overline{D_2}$ and $\overline{D_1} \cap D_2$ are generated, you can use EmptyNFA
above to determine if $L(D_1 \cap \overline{D_2}) = \emptyset$ and $L(\overline{D_1} \cap D_2) =
\emptyset$. Both of these hold iff $(D_1,D_2) \in EqualDFA$.

\item \textbf{2e)} 
For a given CFG,  $G = (V,\Sigma, R, S)$ say that a non-terminal, $N$, is \emph{useful} if there is
a derivation of a string of non-terminals, from $N$.  
The following algorithm will determine if $L(G) = \emptyset$, by marking all the useful
nonterminals.  Then $\sel(G) = \emptyset$ iff $S$ does not end up marked.

\[
\begin{array}{l}
\textrm{Mark each terminal symbol in }G \\
\textrm{Do:} \\
\quad \textrm{Mark any symbol $A$ where there is a rule in $G$ such that }A \rightarrow w_1w_2...w_k \\
\quad \textrm{ where each symbol in $w_1w_2...w_k$ has already been marked.} \\
\textrm{Until no new symbols get marked.} \\
\textrm{If S is not marked: accept. Else: reject.}
\end{array}
\]

\item \textbf{2f)} The following algorithm will determine if $(G,w) \in MemberCFL$, that is,

\begin{equation}
\{\langle G,w\rangle\ | G \in CFG, w \in L(G)\}
\end{equation}

for a given CFG $G = (V, \Sigma, R, S)$ and a given string $w=w_1\cdots w_n$.

\[
\begin{array}{l}
\textrm{Convert $G$ into its Chomsky Normal Form (CNF) representation, $G'$} \\ \\
\textrm{Let $N[i,j]$ be the set of all symbols in $V\cup\Sigma$ that can derive the string
  $w_i \ldots w_j$, with $i \leq j$.} \\[2ex]
\textrm{The following dynamic programming algorithm computes $N[i,i+j]$, by induction on $j$:}\\[2ex]
\textrm{for $i = 1$ to $n$ do}: N[i,i] = \{w_i\} \\
\textrm{for $j = 1$ to $n-1$ do}: \\
\quad \textrm{for $i = 1$ to $n-j$ do}: \\
\quad\quad \textrm{for $k = 1$ to $i+j-1$ do}: \\
\quad\quad\quad \textrm{if there is a rule} A \rightarrow BC \in R \textrm{ where } B \in N[i,k]
\textrm{ and } C \in N[k+1,i+j]: \\
\quad\quad\quad\quad \textrm{then add $A$ to} N[i, i+j]\\
\textrm{if } S \in N[i,n] \textrm{: accept. Else: reject.} 
\end{array}
\]

\end{itemize}


\end{enumerate}

\end{document}
