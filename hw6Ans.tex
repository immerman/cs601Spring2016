\documentclass[12pt]{article}
\usepackage{defs601}                   
\usepackage{algorithmic}
\hwstyle
\begin{document}
\noindent\framebox[\textwidth]{
  {\normalsize CS601:} \hfill {\bf Answers to hw6} \hfill 
  {Spring 2016}}
\addtocounter{section}{1}

% Some time-saving macros:
\newcommand{\re}{\text{r.e.}}
\newcommand{\cre}{\text{co-r.e.}}
\newcommand{\nat}{\mathbb{N}}
\newcommand{\lang}{\mathcal{L}}
\newcommand{\setb}[2]{\left\{ #1 ~|~ #2 \right\}}

\begin{enumerate}

  \item Recall the definition of r.e.:
    $\re = \setb{W_i}{i \in \nat} = \setb{\lang(M_i)}{i \in \nat}$.

    \begin{proof}
      Prove that $\re = \Sigma_1$ (that is, that
      $\re = \setb{S = \setb{n}{\exists x_1.~\phi(n, x_1)}}{\phi \in P}$).
      
      \begin{enumerate}
        \item $\re \subseteq \Sigma_1$:

          Consider $W_i = \lang(M_i) = \setb{n}{M_i(n)=1}$.

          For each r.e. set $W_i$, we need to find a function $\phi_i \in P$.

          If $M_i(n) = 1$, then there exists a \emph{computation} $c$, a
          sequence of TM states which describes the process of $M_i$ computing
          the result $1$ from $n$. The polynomial-time function that
          determines whether $c$ is this computation is $\text{COMP}(i,n,c,1)$.
          Therefore, $W_i = \setb{n}{\exists x_1.~\phi_i(n, x_1)}$, where
          $\phi_i(n, c) = \text{COMP}(i, n, c, 1)$.

        \item $\Sigma_1 \subseteq \re$:

          The following r.e. algorithm defines the set
          $S = \setb{n}{\exists x.~\phi(n, x)}$, for any $\phi \in P$:

          \begin{algorithmic}
            \STATE input $n$
            \FOR{$x \gets 1$ to $\infty$}
              \IF{$\phi(n, x)$}
                \STATE accept
              \ENDIF
            \ENDFOR
          \end{algorithmic}

      \end{enumerate}
    \end{proof}

    {\bf Answer by: Adam Nelson}

  \item $\cre = \Pi_1$ is a simple corollary of $\re = \Sigma_1$: For
    $\cre \subseteq \Pi_1$, note that $\bar{W_i} = \setb{n}
    {\forall c.~\lnot\text{COMP}(i,n,c,1)}$, and, for $\Pi_1 \subseteq \cre$,
    note that changing the if-statement in problem 1 part (b) to
    \begin{algorithmic}
      \IF{$\lnot\phi(n, x)$}
        \STATE reject
      \ENDIF
    \end{algorithmic}
    makes the algorithm co-r.e.

    {\bf Answer by: Adam Nelson}

  \item $\text{EMPTY} = \setb{n}{W_n = \emptyset}$ is in $\Pi_1$:

    $\text{EMPTY} = \setb{n}{\forall x c.~\lnot\text{COMP}(n,x,c,1)}$

    {\bf Answer by: Adam Nelson}

  \item $\text{FINITE} = \setb{n}{W_n~\text{is finite}}$ is in $\Sigma_2$:

    If a TM is finite, then there must be maximum length, $l$, of string that
    it accepts.

    $\text{FINITE} = \setb{n}{\exists l~\forall x c.~\lnot\text{COMP}(n,x,c,1)
      \lor \left|x\right| \le l}$

    {\bf Answer by: Adam Nelson}

  \item $\text{PTM} = \setb{n}{M_n~\text{runs in polynomial time for all inputs}}$
    is in $\Sigma_2$:

    For our $\phi$, we use the function $\text{Halt}(i,x,n)$, which runs $M_i$
    with input $x$ for $n$ cycles, then returns $1$ if it has halted by this
    point, or $0$ otherwise. This runs in polynomial time on $x$ if
    $n = c\left|x\right|^c$ for some constant $c$.

    $\text{PTM} = \setb{n}{\exists c.~\forall x l.~\left|x\right| \ge l 
      \lor \text{Halt}(n,x,cl^c)}$

    {\bf Answer by: Adam Nelson}

\end{enumerate}

\end{document}

