\documentclass[12pt]{article}
\usepackage{defs601}                   
\hwstyle
\begin{document}
\thispagestyle{empty}
\noindent\framebox[\textwidth]{
{\normalsize CS601:} \hfill {\bf Homework 2} \hfill 
{Due in class Tue, Jan. 26, 2016}}
\addtocounter{section}{2}

In Lecture 2 we defined partial recursive functions, total recursive functions, recursive sets,
r.e. sets and co-r.e. sets. We showed that all Turing machines can be listed out as $M_0, M_1, 
\ldots, M_n, \ldots$, where the subscript, $n$, is a natural number, which may also be thought of as a binary
string, $w = \zeta^{-1}(n)$, which is a binary encoding of the the program that $M_n$ runs, i.e.,
$n$ is the program. 

As Turing proved, there is a universal TM, $U$, with the property that for all $i,j\in\N$, $U(i,j) =
M_i(j)$, i.e., the universal machine on input $(i,j)$ exactly simulates $M_i$ on input $j$.

We also defined $W_i = \sel(M_i) = \bigset{w}{M_i(w) = 1}$, i.e., the set accepted by TM $M_i$ is
the $i$th r.e. set.  The set of all r.e. sets is thus $\RE = \set{W_0, W_1, \ldots}$.

We defined the diagonal set, $K = \bigset{i}{i \in W_i}$. We proved that $K$ is r.e., but that it's
complement, $\ov{K}  = \bigset{i}{i \not\in W_i}$ is not r.e.

Please read the notes for lectures 1 and 2 posted on the syllabus page.

Even though the proofs so far are reasonably simple, the concepts are deep and take a while to
absorb.  The goal of this homework is to help you do that.

\begin{enumerate}
\item Let $A_{\text{TM}} = \bigset{(i,j)}{M_i(j) = 1}$.  

\begin{enumerate}
   \item Show directly from the definition of r.e., that $A_{\text{TM}}$ is r.e., i.e., that 
         $p_{A_{\text{TM}}}$ is a partial recursive function.  
   \item Show from the fact that $\ov{K}$ is not r.e., that $A_{\text{TM}}$ is not recursive.  Do
     this by contradiction:  assume that $A_{\text{TM}}$ is recursive, i.e., you have a TM, $M_A$, that
     computes $\chi_{A_{\text{TM}}}$.  Next modify $M_A$ to build a TM that computes $\chi_K$.
\end{enumerate}

\item (Where the name ``recursively enumerable'' comes from.)
  \begin{enumerate}
   \item Show that every finite set, $F$, is r.e.
   \item Suppose that $W_i$ is infinite.  Given $M_i$, show how to build a TM computing a total
     recursive function, $f_i$, such that $f_i$ enumerates $W_i$, i.e., 
\[ W_i \qe \set{f_i(0), f_i(1), f_i(2), \ldots }\]
Can you always construct $f_i$ so that it is a 1:1 function?
  \end{enumerate}

\end{enumerate}

\end{document}

