\documentclass[12pt]{article}
\usepackage{defs601}                   
\usepackage{algorithmic}
\hwstyle
\begin{document}
\noindent\framebox[\textwidth]{
{\normalsize CS601:} \hfill {\bf Answers to hw5} \hfill 
{Spring 2016}}
\addtocounter{section}{1}

\begin{enumerate}
  \item Recall the definition of $F(\dtime{t(n)})$ from the slides for lecture 5.  In particular,
  $F(\p)$ is the set of total functions computable by Turing machines that run in polynomial time.  

Consider the set PTM  
$=\bigset{M_i}{\exists c \forall n(\mbox{ TM $M_i$ runs in time $cn^c$ on all inputs of length $n$}
  )}$. 
That is, PTM is the set of TMs that run in polynomial time.
Give two reductions to show that PTM is neither r.e.\ nor co-r.e.

{\bf Answer by: Lite Ye} \quad {\bf with edits by:}

\begin{proof}
  We are going to show PTM is neither r.e.\ nor co-r.e.\ by showing
  $\mathrm{K}\leq \mathrm{PTM}$ and $\overline{\mathrm{K}}\leq \mathrm{PTM}$.
  
  For any $i$, $f_1(i)$ returns a program that
  \begin{center}
    \begin{algorithmic}
      \STATE Ignore input \IF{$M_i(i)$ halts} \RETURN{1} \ELSE \STATE{Diverge}
      \ENDIF
    \end{algorithmic}
  \end{center}
  This program is in constant time if $i\in \mathrm{K}$. Otherwise it never halts. So $f_1(i)\in
  \mathrm{PTM}$ if and only if $i \in \mathrm{K}$. $\mathrm{K}\leq \mathrm{PTM}$.
  
  For any $i$, $f_2(i)$ returns a program that
  \begin{center}
    \begin{algorithmic}
      \STATE $n \gets$ length of input
      \IF{$M_i(i)$ halts in $n$ steps}
      \STATE Diverge
      \ELSE
      \RETURN 1
      \ENDIF
    \end{algorithmic}
  \end{center}
  For all $i\in \overline{\mathrm{K}}$, $M_i(i)$ never halts. So $f_2(i)$ is in
  linear time and $f_2(i) \in \mathrm{PTM}$. For all $i\not\in
  \overline{\mathrm{K}}$, $M_i(i)$ halts for sufficiently large $n$, then
  $f_2(i)\not\in \mathrm{PTM}$. So $\overline{\mathrm{K}}\leq \mathrm{PTM}$.

  If $\mathrm{PTM}$ is r.e., then $\overline{\mathrm{K}}$ is r.e., which leads
  to contradiction. If $\mathrm{PTM}$ is co-r.e., then $\mathrm{K}$ is co-r.e.
  Thus $\mathrm{K}$ is recursive, also a contradiction.
\end{proof}



\item Show that there is a recursive set of TMs, $R \subseteq$ PTM, such that for all $f \in F(\p)$,
  there exists $M_i \in R$ such that the function computed by $M_i$ is $f$.  [Hint:  construct $R$
    as the set of all TMs which run with an explicit polynomial-time clock which shuts the TM off
    when the clock expires.]

{\bf Answer by: Bruce Spang} \quad {\bf with edits by:}

\begin{proof}
  Let $R$ be the set of turing machines $M_n^c$, for each pair of numbers $(n, c)$. On input $x$, $M_n^c$ will run the $n^{th}$ turing machine for $c|x|^c$ steps on input $x$. If the $n^{th}$ turing machine accepts in $\leq c|x|^c$ steps, $M_n^c$ will accept, otherwise $M_n^c$ will halt and reject.

  Suppose that each $M_n^c$ somehow embeds the pair $(n,c)$ in the machine in a way that is easy to extract (e.g. maybe the state table starts with the pair $(n,c)$, and those states are never used in the transition function). Then $R$ is recursive, since we can extract $(n, c)$ from the definition of the machine, then generate $M_n^c$ and see if the two are same.

  $R \subseteq$ PTM, since the turing machine $M_n^c$ will always halt in polynomial time on any input.

  For each $f \in F(\p)$, there exists an $M_i \in R$ that computes $f$. Since $f$ is in $F(\p)$, $f$ is computable by a turing machine in polynomial time. That is, there exists a machine $M$ and a $c$ such that for all inputs of length $n$, $L(f) = f$ and $M$ always halts in $\leq cn^c$ steps. Since our language generates all such pairs of $M$s and $c$s, each $f$ must have a corresponding pair in the language.
\end{proof}



\end{enumerate}

\end{document}
