\documentclass[12pt]{article}
\usepackage{defs601}                   
\hwstyle
\begin{document}
\thispagestyle{empty}
\noindent\framebox[\textwidth]{
{\normalsize CS601:} \hfill {\bf Homework 10} \hfill 
{Due in class Tue, Mar.\ 22, 2016}}
\addtocounter{section}{9}

As ususal, please ask on Piazza if parts of the following two problems need hints or clarifications.

\begin{enumerate}

\item Prove Proposition 14.2 from Lecture 14:  The set of problems accepted by uniform, bounded-width branching
  programs is contained in  $\nc^1$.  Hint: use an argument similar to Savitch's Theorem and similar
  to the proof that $\nl \subseteq \sac^1$.

\item  %from spring 01
In a Tiling problem, we are
given an integer, $n$, a set of square tile types, $T = \set{t_0, \ldots, t_k}$, together with two
relations $H,V 
\subseteq T \times T$, where $H(t_1,t_2)$ means that $t_2$ can be placed immediately to the right of
$t_1$ and $V(t_1,t_2)$ means that $t_2$ can be placed immediately below $t_1$.  TILING is the
problem of deciding whether there exists an $n\times n$ grid of tiles satisfying the horizontal and
vertical constraints and with $t_0$ in the upper left position.

Show the following:

\begin{enumerate}
\item TILING is NEXPTIME complete.
\item If $n$ is given in unary, then TILING is NP complete.
\item The problem of testing whether for all $n$ there exists such an $n\times n$ tiling is
  co-r.e. complete.  
\end{enumerate}


[Hint: you can encode a Turing machine computation via such tiles.  Let $N$
be a nondeterministic TM.  Consider a set of tiles that are labeled with
the tape symbols from $\Sigma$ plus an additional set of tiles that are
labeled with triples $\angle{q,a,\delta}$ where $q$ is a state, $a\in
\Sigma$ and $\delta\in\set{0,1}$ indicates which nondeterministic choice
the machine will make at this step, plus some additional tiles.  


\begin{center}
\begin{tabular}{|c|c|c|c|c|}\hline
$\triangleright$ & $q_0,a,\delta_0$ & $b;q$ & $c$ & $\sqcup$\\ \hline
$\triangleright$ & $d$ & $q,b,\delta$ & $c;q'$ & $\sqcup$\\ \hline
$\triangleright$ & $d$ & $e$ & $q',c,\delta'$ & $\sqcup$\\ \hline
\end{tabular}
\end{center}

Suppose
for example that choice $\delta$ of $N$ in state $q$ looking at a $b$ is
to write $e$, and go to the right into state $q'$.  Then the tile marked
$\angle{q,b,\delta}$ would have a signal at its bottom indicating that the
next tile should contain symbol $e$ and it could have a signal on its right
indicating that the cell on its right should have a signal on its bottom
indicating that the cell below should be in state $q'$.]
\end{enumerate}

\end{document}

