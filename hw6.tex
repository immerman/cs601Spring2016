\documentclass[12pt]{article}
\usepackage{defs601}                   
\hwstyle
\begin{document}
\thispagestyle{empty}
\noindent\framebox[\textwidth]{
{\normalsize CS601:} \hfill {\bf Homework 6} \hfill 
{Due in class Thr, Feb. 11, 2016}}
\addtocounter{section}{6}

Please finish writing up solutions to any homeworks that you volunteered to do, but haven't done
yet.

Please read the posted notes for lectures 6 and 7 and come in with any questions you may have.


This hw concerns the Arithmetic Hierarchy, which is at the top
of the  World-of-Computability-and-Complexity diagram.  We say that a set
of natural numbers, $S$, is an element of $\Sigma_k$ iff there is a
ptime predicate $\phi$, such that,
\[ S \qe \bigset{n}{\exists x_1\forall x_2 \cdots Q_k
x_k(\phi(n,x_1,\ldots,x_k))},\]
here $Q_k$ is $\forall$ if $k$ is even and $\exists$ if $k$ is odd.
Similarly, $S$ is an element of $\Pi_k$ iff,
\[ S \qe \bigset{n}{\forall x_1\exists x_2 \cdots Q'_k
x_k(\psi(n,x_1,\ldots,x_k))},\]
for some ptime predicate $\psi$.  Here $Q'_k$ is $\forall$ if
$k$ is odd and $\exists$ if $k$ is even. 


\begin{enumerate}
\item Prove that \quad r.e. = $\Sigma_1$.
\item Prove that \quad co-r.e. = $\Pi_1$.

\vspace*{.1in}


Define the Arithmetic Hierarchy to be $\displaystyle
\bigcup_{k=1}^\infty \Sigma_k$.


Classify the following sets by writing a  formula that
places them as low as you can in the arithmetic hierarchy.  
You do not have to prove that they cannot be placed in a lower class,
but do your best.  

For example,  
TOTAL  =  $\bigset{n}{M_n\mbox{ halts on all inputs}}$
is $\Pi_2$ because it can be written as
\[ \mbox{TOTAL} \qe \bigset{n}{\forall x\exists
z(\mbox{COMP}(n,x,L(z),R(z)))}\; .\]
(Here COMP$(n,x,c,y)$ is the ptime predicate meaning that $c$ is a complete halting computation of TM $M_n$
on input $x$ and its output is $y$.)




\item  EMPTY = $\bigset{n}{W_n=\emptyset}$ .
\item FINITE = $\bigset{n}{W_n\mbox{ is finite}}$ 
\item PTM  = $\bigset{M_i}{\exists c \forall n(\mbox{ TM $M_i$ runs in time $cn^c$ on all inputs of length $n$}
  )}$.
\end{enumerate}
\end{document}

