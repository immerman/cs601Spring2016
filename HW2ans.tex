\documentclass[12pt]{article}
\usepackage{defs601}                   
\hwstyle
\begin{document}
\thispagestyle{empty}
\noindent\framebox[\textwidth]{
{\normalsize CS601:} \hfill {\bf Homework 2} \hfill 
{Due in class Tue, Jan. 26, 2016}}
\addtocounter{section}{2}

\begin{enumerate}
\item  Let $A_{\text{TM}} = \bigset{(i,j)}{M_i(j) = 1}$.  

\begin{enumerate}
   \item Show directly from the definition of r.e., that $A_{\text{TM}}$ is r.e., i.e., that 
         $p_{A_{\text{TM}}}$ is a partial recursive function.  
       
{\bf Answer by: Sophie Koffler} \quad {\bf with edits by: Neil}\\
  
         \textbf{Proof}: 
         \\To show that $A_{\text{TM}}$ is r.e., we need to show that there exists a TM, $M_{A}$, that computes  the partial(characteristic) recursive function, $p_{A_{\text{TM}}}$, for $A_{\text{TM}}$. 
         \\We need to show that $\forall w \in \Sigma_{0}^{*} (p_{A_{\text{TM}}}(w) = M_{A}(w))$ where $w$ is the binary encoding of a pair of natural numbers $i$ and $j$, which can be decoded using the left and right pairing functions respectively, as defined in class.
         \\Let $M_{A}$ be the TM that computes $p_{A_{\text{TM}}}(w) = p_{A_{\text{TM}}}(L(w),R(w)) = p_{A_{\text{TM}}}(i,j)$.
		  \begin{center}	 
         \[ p_{A_{\text{TM}}}(i,j) = 
         \left \{
         \begin{tabular}{cc}
         1 & if $(i,j) \in A_{\text{TM}}$  \\
         $\nearrow$ & otherwise \\
         \end{tabular}
         \right.
         \]
	    \end{center}
          More specifically, 
         \begin{center}
          \[ p_{A_{\text{TM}}}(i,j) = 
          \left \{
          \begin{tabular}{cc}
          1 & if $M_{i}(j)=1$  \\
          $\nearrow$  & otherwise \\
          \end{tabular}
          \right.
          \]
        \end{center}
        
         On input X, $M_{A}$ operates as follows:
         \begin{description}
         	\item[1] Verify X is an encoding of a pair of natural numbers, and decode the numbers, i and j using the left and right pairing functions respectively. 
         	\\If X is not valid input reject.  [{\color{blue} Note from Neil: While correct, this is not
                necessary because we assume throughout that every input, $X$, is a natural number
                and thus, since $X = P(L(X),R(X))$, also a pair of natural numbers.}]
         	\item[2] Run TM $M_{i}$ on input $j$:
			         	if ($M_{i}(j)$==1) return 1
			         	else diverge
         \end{description}
        Since we have constructed a TM $M_{A}$ that computes $p_{A_{\text{TM}}}$, we have shown that $A_{\text{TM}}$ is r.e. $\Box$ 
         
   \item Show from the fact that $\ov{K}$ is not r.e., that $A_{\text{TM}}$ is not recursive.  
   
{\bf Answer by: Sophie Koffler} \quad {\bf with edits by: Neil }\\

     \textbf{Proof}: Assume, for the sake of contradiction, that $A_{\text{TM}}$ is recursive.  Therefore we have TM, $M_{A}$, that
     computes $\chi_{A_{\text{TM}}}$, the characteristic function of $A_{\text{TM}}$, which is defined as follows: 
      \begin{center}	 
       	\[ \chi_{A_{\text{TM}}}(i,j) = 
       	\left \{
       	\begin{tabular}{cc}
       	1 & if (i,j) $\in A_{\text{TM}}$  \\
       	0 & otherwise \\
       	\end{tabular}
       	\right.
       	\]
       \end{center}
       
       This means we can construct a TM, $M_{{K}}$, to compute the characteristic function, $\chi_{{K}}$ by modifying $M_{A}$.
       
       Let $M_{{K}}$ be the TM that computes $\chi_{{K}}$.
       
       On input $i$, $M_{{K}}$ operates as follows:
       \begin{description}
       	\item[1] Run TM $M_A$ on input $(i,i)$:
       	\\if ($M_{i}(i)$==1) return 1
       	\\else return 0
       \end{description}
       
       Therefore, $M_{{K}}$, calculates $\chi_{{K}}$ where
       
      \begin{center}	 
      	\[ \chi_{{K}}(i) = 
      	\left \{
      	\begin{tabular}{cc}
      	1 & if $(i,i) \in A_{\text{TM}}$ \\
      	0 & otherwise \\
      	\end{tabular}
      	\right. 
      	\]
      \end{center}
      
      Which is equal to 
      \begin{center}	 
      	\[ \chi_{{K}}(i,i) = 
      	\left \{
      	\begin{tabular}{cc}
      	1 & $M_{i}(i)=1$ \\
      	0 & otherwise \\
      	\end{tabular}
      	\right. 
      	\]
      \end{center}
       
      By constructing TM $M_{A}$ that
      computes $\chi_{A_{\text{TM}}}$, we have shown the existence of a TM $M_{{K}}$ that calculates $\chi_{{K}}$. 
      
      Therefore $K$ is recursive. But we previously proved that $K$ is not recursive.

      $\Rightarrow\Leftarrow$


      Therefore,  $A_{\text{TM}}$ is not recursive. $\Box$
     
\end{enumerate}

\item (Where the name ``recursively enumerable'' comes from.)
  \begin{enumerate}
   \item Show that every finite set, $F$, is r.e.
   
    \textbf{Proof}: Let $M_\text{F}$ be the TM that computes the characteristic function $\chi_F$.
    
    $M_\text{F}$ on input $w$ operates as follows: 
    \\If $w \in F$: return 1
    \\		 else return 0
    
   The TM, $M_{\text{F}}$ keeps a table for all the elements of the finite set, $F$.  It accepts
   everything in its table and rejects otherwise.

    Therefore, we have shown that every finite set, $F$, is recursive. 
    Thus every finite set is also r.e. 
$\Box$
    
    
   \item Suppose that $W_i$ is infinite.  Given $M_i$, show how to build a TM computing a total
     recursive function, $f_i$, such that $f_i$ enumerates $W_i$, i.e., 
\[ W_i \qe \set{f_i(0), f_i(1), f_i(2), \ldots }\]

		\textbf{Proof}: Since we know that $W_i = \sel(M_i)$ is infinite, find some element,
                $a_0 \in W_i$.

To compute $f_i(x)$ we do the following:  Simulate $M_i(L(x))$ for $R(x)$ steps.  If in this amount
of time, $M_i(L(x))$ halts and outputs $1$, then $\Return(L(x))$, \Else $\Return(a_0)$.

Observe that $f_i$ is total -- in fact in runs in linear time -- and $f_i(\N) = W_i$.  The range of
$f_i$ contains only elements of $W_i$.  If $a\in W_i$, then $M_i(a) = 1$.  Let $t_a$ be the number
of steps that $M_i(a)$ takes.  Then $f_i(P(a,t_a))= a$.


\vspace*{.1in}



Can you always construct $f_i$ so that it is a 1:1 function?

Yes.  To do this, on input $x$, we must dovetail $M_i$ on all possible inputs for longer and longer
amounts of time, keeping track of each new element accepted: $a_0, a_1, \ldots, a_x$.  When we reach
the $x$th accepted element, $a_x$, we return that element and halt.
  \end{enumerate}

\end{enumerate}

\end{document}
